\chapter*{Introducción}
\addcontentsline{toc}{chapter}{Introducción}

En la actualidad, el campo de la criptografía ha experimentado una revolución significativa gracias 
a los avances en inteligencia artificial y teoría del caos. Este trabajo se centra en el desarrollo 
y análisis de un modelo de criptosistema innovador que integra redes neuronales y atractores caóticos, 
representando un enfoque novedoso en la seguridad de la información.

Los criptosistemas tradicionales, aunque robustos, enfrentan desafíos crecientes debido a la evolución 
de las capacidades computacionales y a las nuevas demandas de seguridad en la era digital. Frente a este 
panorama, las redes neuronales ofrecen una capacidad de aprendizaje y adaptación única, mientras que la 
teoría de los atractores caóticos aporta un nivel de imprevisibilidad y complejidad que potencialmente 
puede fortalecer los mecanismos de cifrado.

Este documento explora la fusión de estas dos poderosas herramientas. Se diseña y simula un criptosistema 
que utiliza redes neuronales para generar claves dinámicas y atractores caóticos para introducir patrones 
de comportamiento no lineales e impredecibles en el proceso de cifrado. La investigación se enfoca en evaluar 
la eficacia, seguridad y viabilidad de este modelo en diversos escenarios, comparándolo con sistemas 
criptográficos convencionales.

Además, se analizan las implicaciones teóricas y prácticas de integrar la inteligencia artificial y la 
teoría del caos en la criptografía. Se discuten los desafíos y las oportunidades que este enfoque presenta, 
tanto en términos de seguridad de la información como en su aplicación en campos como la comunicación segura, 
la protección de datos y la ciberseguridad.

Por último, este trabajo no solo busca aportar un modelo teórico innovador, sino también sentar las bases para 
futuras investigaciones y desarrollos prácticos en la intersección de la criptografía, la inteligencia artificial 
y la teoría del caos, abriendo nuevas vías para la seguridad de la información en el mundo digital.