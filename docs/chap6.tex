\chapter{Metodología}

Para el desarrollo del presente proyecto se trabajarán diferentes fases basadas en cada uno de los objetivos a ejecutar.

\section {Fase I}

Para mejorar un proceso ya implementado, lo primero que se debe plantear es una investigación rigurosa con respecto a todas las variables tenidas en cuenta en dicho proceso, de manera que quede cubierta cada una de las posibilidades de aquello que se debe y puede mejorar; por este motivo, lo primero que se efectuará será el estudio y evaluación de cada uno de los procesos a optimizar, para ello se estudiarán las encuestas realizadas por los organismos de tránsito existentes, si la información aportada no es suficiente para el proyecto,se realizarán encuestas para cumplir con el objetivo, así mismo se evaluarán todos los índices posibles y apropiados con respecto a la movilidad evaluada, métodos de desarrollo y las tecnologías que se pueden implementar.
\subsection{Investigación exhaustiva sobre las ITS' }

Para reconocer la forma en que operan los semáforos en la actualidad se hace necesario, estudiar su evolución y en este concepto entra todo aquello relacionado a las TIC's.
Para esta actividad se estudiará acerca de lo relacionado a tecnologías inteligentes enfocadas al transporte; indagando en los lugares donde se han puesto en marcha estos sistemas y realizando análisis de esta información.  

%Lo primero que se hará sera analizar todas las variables que se pueden generar con respecto a los sistemas de transito, se evaluaran las apropiadas y posibles encuestas, índices, estadísticas y opiniones respecto a este ámbito, con la finalidad de obtener la mayor cantidad de información posible referente al funcionamiento de dichos sistemas.
\subsection{Determinación del modo de encriptación de la información}
Puesto que se esta manejando información a la que no cualquiera debe tener acceso se hace necesario encriptar los datos a enviar; para esto se realizará un estudio de las formas de proteger información que existen en la actualidad para seleccionar el más apropiado.

\subsection{Preparación y análisis teórico}

Para desarrollar una idea, es necesario conocer a fondo todas las maneras en que se puede abordar dicha idea, por lo tanto, esta fase del proyecto sera muy importante, puesto que se llevará a cabo una profunda documentación con respecto a todos los métodos que pueden ser implementados.

Se evaluará cada unos de los protocolos que pueden ser implementados en los procesos de las ITS'; centrándose en los estándares NTCIP, SCATS y OCIT, dado que estos han sido utilizados en el país; en esta etapa se realizarán análisis de factores como información al respecto, tendencia de implementación y la justificación del uso de los ya mencionados estándares, en pocas palabras realizar un cuadro comparativo entre pros y con-tras de cada estándar para así avanzar descartar ó establecer el indicado.
    
\section {Fase II}

Con respecto a la movilidad del medio a mejorar; teniendo en cuenta los estudios realizados anteriormente, se escogerán los métodos apropiados de monitoreo a implementar para que el sistema tome las decisiones apropiadas en tiempo real, con respecto a las finalidades deseadas.

\subsection{Elección de métodos}

Elegir uno de los protocolos anteriormente enunciados, con respecto a las necesidades planteadas y los alcances de los empleadores, de manera apropiada, es una de las decisiones mas importantes a tomar, puesto que esta es la finalidad del objetivo general del proyecto, de manera que no se escatimara en recursos al momento de efectuar esta fase. Se estudiarán dichos protocolos los mas profundamente posible para definir cual sera implementado, realizando las respectivas pruebas.


\subsection{Implementación del protocolo}

Ya escogido y probado el protocolo a ejecutar, sera implementado y asi, se adaptara este a la antena GPRS que se utilizará, actualizando el controlador que actualmente maneja la compañía .

\section {Fase III}

\subsection{Modificación software }

Puesto que solo se trabajará sobre la comunicación entre el módulo y la central de operación es necesario modificar el software de la central para que tome la información procedente de la antena GPRS y le desarrolle un tratamiento tal que muestre en pantalla justo lo que se envió desde el módulo. 

%Con el fin de mejorar los procesos ya planteados, se procederá a implementar una plataforma web para el monitoreo de los controladores enunciados anteriormente, mediante protocolos y tecnologías  óptimas.


%Se elegirá de manera correcta los sistemas apropiados, para disminuir los tiempos de espera. Se desarrollaran herramientas mas amigables que faciliten el control y la supervisión de los sistemas de transito. 

\section {Fase IV}

Ya habiendo sido implementados todos los conceptos de desarrollo general del proyecto, se procederá a incorporar todos los módulos en uno solo.

\subsection{Verificación y ajuste del modulo RF}
Dado que este módulo ya existe, solo se plantea esta actividad para corroborar su funcionalidad, en caso de algún error o falla se ajustará a las necesidades del proyecto.

\subsection{Verificación y ajuste de sistema de potencia}

Uno de los aspectos fuertes del sistema radica en que la fuente de alimentación(voltaje) se realiza de forma autónoma, es decir se  utiliza paneles fotovoltaicos y se realiza toda una etapa de potencia para suministrar la energía eléctrica necesaria para el funcionamiento de los demás módulos; en este proceso solo se verificara que su funcionamiento sea el adecuado y en caso de algún error o falla se ajustará a las necesidades de este proyecto.


\subsection{Verificación y ajuste del modulo LAN}

En esta etapa se corroborará que la comunicación entre los dispositivos XBEE funcione como han venido trabajando y al igual que en las fases anteriores en caso tal de encontrar un error se procede a repararlo.

\section {Fase V}

Como todo proyecto lo demanda, la fase siguiente corresponde a la materialización de todo diseño efectuado anteriormente. 

\subsection{Programación inicial}

En esta etapa se configurará el primer plan de trabajo del semáforo, a través de la comunicación preparada entre el modulo y la central de control de la compañía.
\subsection{Verificación conexión con central existente }

Dado que uno de los objetivos es comprobar el funcionamiento del protocolo, se realizarán pruebas de conexión, es decir lograr enviar y recibir información desde el módulo hasta una central que ya posee el protocolo seleccionado en etapas anteriores.
Por lo dicho anteriormente la cuestión radica en realizar el mismo tipo de funcionamiento que se opera entre el módulo y la central de la compañía sino ahora entre el modulo y una central mucho más grande.

\subsection{Pruebas de calidad del sistema}


Ya instalado el sistema ITS, se harán las respectivas pruebas de calidad. Esperando que del sistema responda de manera óptima para concluir con el trabajo de manera exitosa y satisfactoria para todas las partes involucradas.